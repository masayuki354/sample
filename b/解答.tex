\documentclass[../main]{subfiles}
\begin{document}
\subsection*{解答}
(1)$
\frac{2}{x} + \frac{1}{y}  = \frac{1}{4}より、
xy -4x -8y = 0であるから、\\
(x-8)(y-4)  =  32 \\
x,yは正の整数で、x-8>-8,y-4>-4となるので、\\
(x-8,y-4) = (1,32),(2,16),(4,8),(8,4),(16,2),(32,1)\\
即ち、(x,y) = (9,36,),(10,20),(12,12),(24,6),(40,5)\\
\\
(2)\frac{2}{x} + \frac{1}{y} = \frac{1}{p} より、
xy - 2py-px = 0であるから、 \\
(x-2p)(y-p)=0\\
x,yは正の整数より、x-2p>-2p,y-p>-p、pは3以上の素数なので\\
(x-2p,y-p) = (1,2p^2),(2,p^2),(p,2p),(2p,p),(p^2,2),(2p^2,1)\\
ここで、X = 2x+3y-7pとすると、2x+3y-7p=2(x-2p)+3(y-p)であるから、\\
(x-2p),(y-p)の値からXの最小値を求めることができる。\\
上記の候補をXに代入していくと、\\
X=2+6p^2,4+3p^2,8p,7p,2p^2+6,4p^2+3\\
pは3以上の素数なので\\
2+6p^2>2p^2+6,4p^2+3>4+3p^2,8p>7pは明らか。\\
(4+3p^2)-(2p^2+6) = p^2 - 2>0 \quad すなわち4+3p^2>2p^2+6\\
(2p^2+6)-7p = (2p-3)(p-2)>0 \quad よって2p^2+6>7p\\
したがって、(x-2p,y-p) =(2p,p)の時にX=7pで最小値をとる。\\
2x+3y= X+7pより
求める最小値は(x,y) =(4p,2p)\\
$
\end{document}